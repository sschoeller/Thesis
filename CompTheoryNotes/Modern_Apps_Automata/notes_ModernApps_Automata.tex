% ----------------------------------------------------------------------------
% METADATA
% Notes by: Scott Schoeller (sschoellerSTEM)
% Book: Modern Applications of Automata Theory
% Editors: D'Souza, D. andShankar, P.
% Year: 2012
% ----------------------------------------------------------------------------
\documentclass[12pt]{article}

\usepackage[document]{ragged2e} % left aligns text
\usepackage{amssymb}
\usepackage{standalone} % for includes
\usepackage{graphicx}
\usepackage{float} % aligns figures


\begin{document}

\pagenumbering{gobble} % toc has no page number
\tableofcontents

\newpage %
\pagenumbering{arabic}

\section{Chapter 1}
\ \\
\textbf{\S{1.1}}\\
finite automaton - finite no. of states, like CPU\\
Only current state can be viewed.\\
\ \\
Notation\\
$A$ finite alphabet\\
Elements of $A$: letters\\
Finite seq. of letters is a word.\\
Empty sequence/word symbolized by $\epsilon$, $\Lambda$, or 1\\
Set of nonempty words symbolized by $A^{+}$\\
Concat non-commutative\\
$|uv| = |u| + |v|$\\
u$\epsilon$ = $\epsilon$u = u
\newpage %

\textbf{\S{1.2.1}}\\
Concat product: $KL = \{uv | u \in K$ and $v \in L\}$\\
Power notation: $L^{n}$, where $L^{0} = \{ \epsilon \}$\\
\ \\
morphism\\

For alphabets $A$ and $B$, a morphism from $A^{*}$ to $B^{*}$ is a mapping.\\ 
$\phi: A^{*} \to B^{*}$ s.t. :\\
$\phi(\epsilon) = \epsilon$\\
$\forall u,v \in A^{*}, \phi(u)\phi(v)$\\
Rational (Regular) Languages: Rat$A^{*}$ is least class of languages over the alphabet, A, s.t. :\\
1. the langauges $\emptyset$ and $\{ a \}$ are rational $\forall a \in A$,\\
2. if $K$ and $L$ are rational languages, then $K \cup L, KL$ and $L^{*}$ are also rational.\\
\ \\
Extended rational operations: rational ops + intersection, compliment, morphic image (?)\\
Class of extended rational languages over $A$ is X-Rat$A^{*}$  

\newpage %

\textbf{\S{1.2.2}}\\
(Finite State) Automata DEF: $A = (Q, T, I,F)$\\

Complete Automata: Finite state automata complete if $\forall \ q \in Q, a \in A, \exists$ at least one transition $(q, q q^{'})$ (NOTE: edge exists starting from each state)\\
$\epsilon$-automata: allow transitions on the empty word. Every $\epsilon$-automaton equivalent to an ordinary automaton.

\textbf{\S{1.2.3}}\\
``Let A be a deterministic automaton and let $w$ be a word."\\
``(1)For each state $q$ of A, $\exists$ at most one path labeled $w$ starting at $q$."\\
(2)If $w = L(A)$, then $w$ labels exactly one successful path.\\
transition function $\delta$ maps $Q \times A \to Q$; $\delta$ maps $(q, a) \in Q \times A$ to the state $q^{'}$ s.t. $(q,a,q^{'}) \in T$
\section{Chapter 3}

\section{Chapter 9}

\end{document}
