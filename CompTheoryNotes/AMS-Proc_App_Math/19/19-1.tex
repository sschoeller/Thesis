% ----------------------------------------------------
% METADATA
% Notes by: Scott Schoeller (sschoellerSTEM)
% Article Author: Robinsons, J.A.
% Proceedings Symposia Applied Math
% Year: 1967
% ----------------------------------------------------

\section{A review of automatic theorem proving}

\subsection{Introduction}
-- Proving \textit{B} follows from \textit{A}: conditions of \textbf{first-order predicate calculus}\\
-- Single non-empty collection \textit{D}: \textbf{universe of discourse}\\
-- Relation symbols on $D^{n}$ (\textit{n} is degree of relation)\\
-- Relation symbols with $n=0$, truth value already determined\\
-- Variables, arbitrary values of \textit{D}\\
-- Terms, from functions composed with their arguments\\
-- Atomic formulae or atoms: relation symbol followed by a parenthesized list of terms; either T or F\\
-- \textit{B} is said to follow from \textit{A} iff no way for \textit{D} to be chosen, so that $(A \land \lnot B)$ is T.\\
-- Statement can't be T is \textbf{unsatifiable}.  If choice that is T, satisfiable.\\

\subsection{A quick summary of the background theory}
-- Show statment, \textit{S}, containing $n \ge 0$ vars, is unsatisfiable \textit{S} is treated as a combination of atoms. \\
-- $S(x_1, ..., X_n)$, show no interpretation of \textit{vocabularly} $R_1 ..., R_k$, $f_1, ...f_m$ of \textit{S} which makes \textit{S} T.\\

