% --------------------------------------------------------------
% METADATA
% Notes by: Scott Schoeller (sschoellerSTEM)
% Author: Rabin, M.O.
% Proceedings Symposia Applied Math
% Year: 1967
% --------------------------------------------------------------

\renewcommand{\thesubsection}{\Roman{subsection}}

\section{MATHEMATICAL THEORY OF AUTOMATA}

\subsection{FINITE AUTOMATA}
word on $\Sigma$
for a finite sequence $x=\sigma_{0}\sigma_{1}...\sigma_{n-1}, \sigma_{i} \in \Sigma$\\
The length $l(x)$ is the no. of elements in \textit{x}\\
$\Lambda$ is the empty word.\\
"$\Sigma^{*}$ is a free semigroup under" concatenation\\
$\sigma \in \Sigma$ are "free generators" of this semi-group\\
For events (or subsets) $A \subseteq \Sigma^{*}, B \subseteq \Sigma^{*}$\\
$A^{*} = A^{0} \cup A ...$\\

\textbf{DEFINITION 1}:
"A \textit{finite automata (f.a.)} over $\Sigma$ is a system $\mathfrak{U} = < S, M, s_0, F>$ 
where \textit{S} is a finite set (the set of \textit{states}), $M: S \times \Sigma \to S$ (the \textit{table of transitions} of $\mathfrak{U}$) 
$s_0 \in S$ ($s_0$ is the \textit{initial state}), $F \subseteq S$ (\textit{F} is the set of \textit{designated} final states)."\\

\textit{M} "can be uniquely extended to" $M^{*}: S \times \Sigma^{*} \to S$; $M^{*}(s, \Lambda) = s, s \in S$\\
$M^{*}(s, x\sigma) = M(M^{*}(s,x), \sigma, s \in S, \sigma \in \Sigma, x \in \Sigma^{*}$\\

\textit{M} is the state-transition function with respect to  $\mathfrak{U}$.\\
In a similar manner, $M^{*}$ describes state-transitions under input \textit{words} $x \in \Sigma^{*}$.\\

\textbf{DEFINITION 2}: 
"The set $T(\mathfrak{U}) = \{ x | x \in \Sigma^{*}, M^{*}(s_0, x) \in F \}$ \\
is the \textit{set} (event) \textit{defined} by $\mathfrak{U}$. A set $A \subseteq \Sigma^{*}$ is called a \textit{regular event} if for some finite automaton $\mathfrak{U}, A = T(\mathfrak{U})$."

\subsubsection{F.a. mappings}
Little distinction between automata in DEFINITION 1 and those that "yield a mapping from input sequences to output sequences".\\
Associate "state $s \in F$ the output 1 and with each $s \in S - F$ the output 0".\\
Associate $\mathfrak{U}$ the mapping $T:\Sigma^{*} \to \{ 0, 1 \}^{*}$ s.t. $T(\sigma_{0} \sigma_{1}...\sigma_{n-1})$,\\
Where for $ i \in [0, n-1], \tau_i = 1$ "if $M^{*}(s_0, \sigma_0 ... \sigma_i) \in F$ and and $r_i = 0$ otherwise."\\
 \newpage
Generalizations possible:\\
1. Partitioning \textit{S} into disjoint "union of \textit{k} sets", where $S= F_0 \cup ... \cup F_{k-1}$; mapping $T:\Sigma^{*} \to { \{ 0, 1, ..., k-1 \} }^{*}$
into words from a \textit{k}-letter alphabet.\\
2. Output of $F_i, i \in,[0, k-1]$ as a word, $w_i$ on alphabet $\Omega$.  Mapping is $T:\Sigma^{*} \to \Omega^{*}$ ("not even length preserving")\\
3. Output as function $f(s, \sigma)$ with respect to the current state and current input of $\mathfrak{U}$

\subsubsection{Nonderministic automata}
$\mathfrak{U}$ in state \textit{s} with input $\sigma$; can go into any one of a number of states $s' \in S'$, where $S \subset S$ is a set depending on $s$ and $\sigma$.\\
As a result of DEFINITION 1: $M: S \times \Sigma \to P(S)$; "$P(S)$ is the power set of $S$ and $s_0$ is replaced by a set $S_0 \subseteq S$ of initial states."\\
If $x=\sigma_0 \sigma_1 ... \sigma_{n-1}$, the sequence of states ($s_0, s_1, ..., s_n$) is \textbf{compatible} w/ x "if $s_0 \in S_0$" and for $i \in [0, n-1]$.\\
$\mathfrak{U}$ accepts $x$ if for some sequence of states through $s_n$ is compatible w/ $x$, $s_n \in F$.\\
The set $T(\mathfrak{U})$ consists of all words accepted by $\mathfrak{U}$.\\
\\
THEOREM 1: "For every nonderministic automata $\mathfrak{U}$ there exists a f.a. $\mathfrak{B}$" s.t. "$T(\mathfrak{U}) = T(\mathfrak{B})$. 
If $\mathfrak{U}$ has n states, then $\mathfrak{B}$" has less than "$2^n$ states." % correction to wording made here


\subsubsection{Regular expressions and events}
Let $Y_1, Y_2, ...$ be vars "ranging over subsets of $\Sigma^{*}$  The set of regular terms in $Y_1, Y_2, ...$ , $\mathbf{R}$ "is the smallest set satisfying the conditions":\\
1. $Y_n \in \mathbf{R}$, where $n \in [1, \infty)$\\
2. if $R_1 \in \mathbf{R}, R_2 \in \mathbf{R}$ then $R_1 \cup R_2 \in \mathbf{R}, (R_1 R_2) \in \mathbf{R}, R_{1}^{*} \in \mathbf{R}$\\
Every element of $\textbf{R}$ is called a regular expression $R(\sigma_1, ..., \sigma_n)$, which is a singleton set.\\
A regular expression a way of expressing subsets of $\Sigma^*$\\
THEOREM 2. 
"A set $T \subseteq \Sigma^*$ is f.a. definable (regular)" iff $\exists R(\sigma_1, ..., \sigma_n)$ where $\sigma_i \in \Sigma, i \in [1,k]$ s.t. $T = R(\sigma_1, ..., \sigma_k)$\\
Every regular event is representable by a regular expression.\\  
Two regular terms $R(Y_1, ..., Y_n)$ and $Q(Y_1, ..., Y_n)$ are called equivalent if $\forall A_1 \subseteq \Sigma^*, ...,  \forall A_n \subseteq \Sigma^*$,\\
$R(A_1, ..., A_n) = Q(A_1, ..., A_n)$\\
\\
The "equivlance of regular terms is effectively solvable."\\
If fixed letters of $\Sigma$ are part of "term formulation", $Y_n \in \mathbf{R^{'}}, \sigma \in \mathbf{R^{'}}, n = 1,2, ... \sigma \in \Sigma$.\\
Regular expressions yield f.a. definable sets.

\subsubsection{Algebraization of f.a.}
$E$ is an equiv. relation on $\Sigma^{*}$ is a right-invariant if $xEy$ implies $xzEyz \forall x, y, z \in \Sigma^{*}$\\
left-invariant(?)\\
congruence on $\Sigma^{*}$: given a set T that is a a subset of $\Sigma^{*}$, one can define two relations...\\
\\
THEOREM 3.\\
$T \subseteq \Sigma^{*}$ is regular iff it's "the union of equiv. classes of a right-invariant relation $E$ (on $\Sigma^{*}$) with finite index."  $T$ is regular iff "$E_{T}$ has a finite index."\\
$\exists \mathfrak{U}$ "with index($E_{T}$) states" s.t. $T = T(\mathfrak{U})$. "No automaton with fewer than index($E_{T}$) states defines $T$."\\
\\
THEOREM 3 "is very useful in show that certain sets $T \subseteq \Sigma^{*}$ are or are not regular."\\
\\
THEOREM 4."A set $T \subseteq \Sigma^{*}$ is regular" iff "it is a union of equivalence classes of a congruence $\equiv$ (on $\Sigma^{*}$ with finite index."  "...$T$ is regular" iff "index($\equiv_{T}$) is finite."\\
"Given a congruence relation" on $\Sigma^{*}$, the partitioned set $\Sigma^{*} / \equiv$ of equivalence classes of $\Sigma^{*}$ with respect to $\equiv$, into semi-group so that mapping $\phi_0 : x \to [x]_{\equiv}$  of ea. $x \in \Sigma^{*}$ into equiv. class is a homomorphism of $\Sigma^{*}$ onto $\Sigma^{*} / \equiv$.\\
Conversely, to ea. homomorphism $\phi_1 : \Sigma^{*} \to M$ of $\Sigma^{*}$ onto semi-group, $M$ - corresponding congruence $\equiv$ def by $x \equiv y$ iff $\phi_1 (x) = \phi_1 (y)$.\\
\\
THEOREM 4 (restated).\\
"A set $T \subseteq \Sigma^{*}$ is regular" iff $\exists M, H \subseteq M$ and a homomorphism $\phi : \Sigma^{*} \to M$ s.t. $T = \phi^{-1} (H)$, where M is a finite semigroup.\\
\\
Associate with ea. $\sigma \in \Sigma$ a function $f_{\sigma}$.  The system $<A, a_0, f_{\sigma}>_{\sigma \in \Sigma}$ is an \textit{algebra of type} $\Sigma$ if $a_0 \in A$

\subsection{PROBALISTIC AUTOMATA}
Not relevant to problem.

\subsection{TREE-AUTOMATA}