% -----------------------------------------------------------------------------------------
% METADATA
% Notes by: Scott Schoeller (sschoellerSTEM)
% "Testing and Generating Infinite Sequences by a Finite Automaton"
% Author: Robert McNaughton
% Journal: Information and Control
% Year: 1966
% -----------------------------------------------------------------------------------------
\documentclass[12pt]{article}
\usepackage{geometry} % see geometry.pdf on how to lay out the page. There's lots.
%\geometry{a4paper} % or letter or a5paper or ... etc
% \geometry{landscape} % rotated page geometry
\usepackage{amssymb}
% See the ``Article customise'' template for come common customisations

\title{Notes on:\\"Testing and Generating Infinite Sequences by a Finite Automaton"}
\author{}
\date{} % delete this line to display the current date

%%% BEGIN DOCUMENT
\begin{document}

\maketitle

Kleene-Myhill Theorem is stated here as "...class of regular events equals the class of finite-state events."  Regular event results from a regular expression; finite-state event "given by a state graph."\\
A way for testing if a word has membership in the event.  This "...theorem implies events generated by finite state mechanisms" same as tested by those mechanisms.\\
\underline{Definitions}\\ 
$\omega$-event: "...set of infinite sequences from some finite input alphabet"; regular expressions extended via the following...\\
$\alpha$ is a "nonempty event not containing" $\epsilon$ (null word). $\alpha^{\omega}$ is the set of infinite seq.  Alternatively, "$\omega$-concatenation of words from $\alpha$" $\alpha$ is going from left to right.  Union of two-$\omega$ events is also an $\omega$-event.  A $\omega$-event is \textit{regular} if $\exists \ \alpha_{1}, ..., \alpha_{n}, \beta_{1}, \beta_{2}, ... \beta_{n}$ s.t. $E = \alpha_{1}\beta_{1}^{\omega} \ \mathbf{\cup} ... \mathbf{\cup} \ \alpha_{n}\beta_{n}^{\omega}$. ($E$ is defined as an $\omega$-event.)\\
"An $\omega$-event is \textit{finite-state} if there is a finite automaton...and a subclass $\{ \pi_{1}, ... \pi_{m} \}$ of the class of all nonempty subset of states of the automaton..." s.t. "for any infinite" seq. "$S$, whose terms are from..." $\Sigma, S \in E$ iff "the precise set of states that the automaton assumes infinitely often when given $S$ as an input" seq...."is one of the sets $\pi_{1}, ..., \pi_{m}$\\
\underline{Theorem (proof in paper)}\\
"An $\omega$-even is regular" iff "...it is finite state. (Also, given the event characterized in one way, the other kind of characterization is effectively determined.)"

\end{document}