% ------------------------------------------------------------------------------------------------------------
% METADATA
% Notes by: Scott Schoeller (sschoellerSTEM)
% "AN n log n ALGORITHM FOR MINIMIZING STATES IN A FINITE AUTOMATON"
% Author: John Hopcroft
% Book: Theory of Machines and Computation (Proceedings)
% Year: 1971
% ------------------------------------------------------------------------------------------------------------
\documentclass[12pt]{article}
\usepackage{geometry} % see geometry.pdf on how to lay out the page. There's lots.
%\geometry{a4paper} % or letter or a5paper or ... etc
% \geometry{landscape} % rotated page geometry
\usepackage{amssymb}
% See the ``Article customise'' template for come common customisations

\title{Notes on:\\"AN n log n ALGORITHM FOR MINIMIZING STATES IN A FINITE AUTOMATON"}
\author{}
\date{} % delete this line to display the current date

%%% BEGIN DOCUMENT
\begin{document}

\maketitle

$A = (S, I, \delta, F)$ is a finite automata.  S and I are finite, $\delta$ is a mapping from $S X I$ into $S$ and $F \subseteq S$.\\
\underline{Algorithm}\\
1. For ea. $s \in S$ and ea. $a \in I$ construct $\delta^{-1}(s,a) = \{ t | \delta(t,a) = s \}$\\
2. Construct $B(1) = F$, $B(2) = S - F$ and for ea. $a  \in I$ and $1 \le i \le 2$ construct\\
\\ $\hat{B}(B(i), a) = \{ s | s \in B(i)$ and $\delta^{-1}(s,a) \ne \emptyset$   \}\\
3. Set $k = 3$\\
4. For ea. $a \in I$ construct\\
\\  $L(a) = {1}$ if  $| \hat{B}(B(1), a) | \le | \hat{B}(B(2), a) |$\\
 $L(a) = {2}$ otherwise.\\
5. Select $a \in I$ and $i \in L(a)$.  The algorithm terminates when $L(a) = \emptyset$.\\
6. Delete $i$ from $L(a)$.\\
7. For each $j < k$ s.t. there exists $t \in B(j)$ with $\delta(t,a) \in \hat{B}(B(i),a)$, perform steps 7a, 7b, 7c and 7d.\\
\\
7a. Partition B(j) into $B^{'} = \{t | \delta(t,a) \in \hat{B}(B(i), a)$ and $B^{"} = B^{"} = B(j) - B^{'}(j) \}$\\ 
7b. Replace $B(j) by B^{'}(j)$. Construct the corresponding $\bar{B}(B(j), a)$. and $\hat{B}(B(k), a)$ for ea. $a \in I$.\\ 
7c.For ea. $a \in I$\\
\\
$L(a) = L(a) \cup {j}$ if $j$ not in $L(a)$ and $0 < ...$ 
\end{document}